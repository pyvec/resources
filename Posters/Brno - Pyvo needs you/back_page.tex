\documentclass[8pt]{article}
\usepackage[a5paper]{geometry}
\usepackage[pdfborder={0 0 0}]{hyperref}
\usepackage{titlesec}
\usepackage{titling}
\usepackage{fontspec}
\usepackage{enumitem}
\usepackage{tgpagella}
\usepackage{microtype}

\newfontfamily\headingfont[]{Bree Serif}
\titleformat*{\section}{\LARGE\headingfont}
\titleformat*{\subsection}{\Large\headingfont}
\titleformat*{\subsubsection}{\large\headingfont}
\renewcommand{\maketitlehooka}{\headingfont}

\renewcommand{\baselinestretch}{1.25} 

\setmainfont[Numbers=OldStyle]{TeX Gyre Pagella}

\setlength{\parindent}{10pt} 

\begin{document}
\fontspec{Oxygen}
\thispagestyle{empty}

\section*{Pomoz organizovat Pyvo}

Organizace Pyva je v zásadě jednoduchá, ale po letech se z~toho stala
zaběhlá, šedá rutina.
Chceme novou krev, která časem přijde s~novými nápady!

Chceš-li se zapojit, ozvi se na brno-pyvo@googlegroups.com (či přímo aktuálním
organizátorům).
Přidáme tě na organizační chat (Slack), kde se to hemží lidmi, kteří
jsou ochotní s~přípravami pomoct. Klidně jen občas.

Vždycky je potřeba udělat následující věci – a nebo se s~někým domluvit, aby je
udělal:

\begin{itemize}[leftmargin=*]
\item
    Zarezervovat hospodu (či jiné místo), nejlépe s Wi-Fi,
\item
    Vytvořit událost na Facebooku a pozvat ostatní organizátory jako spolupořadatele,
\item
    Poslat pull request s informacemi na github.com/pyvec/pyvo-data
\end{itemize}

\bigskip

\noindent
Jako bonus se toho dá udělat víc. Nás za ty roky napadlo například:

\begin{itemize}[leftmargin=*]
\item
    Vybrat nějaké téma,
\item
    Sehnat přednášející (a~projektor a~plátno, případně i mikrofon),
\item
    Moderovat,
\item
    Sehnat sponzory jídla a pití,
\item
    Udělat to celé úplně jinak – pozvat lidi do parku, udělat grilovačku, jet hromadně do Bratislavy...
\end{itemize}

\noindent
Další tipy najdeš na http://pyvec-guide.readthedocs.io

\end{document}
